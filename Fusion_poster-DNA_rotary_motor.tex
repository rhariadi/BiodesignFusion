% vim:fdm=marker
\documentclass[final,t]{beamer}

% Packages and Title %{{{
%\usepackage[orientation=landscape,size=a0,scale=1.4,debug]{beamerposter}
\usepackage[orientation=landscape,size=custom,width=148.625,height=105.125,scale=1.75,debug]{beamerposter}
\usetheme{FusionASU}
\usepackage{amssymb}
\usepackage{amsfonts}
\usepackage{amsmath}
\usepackage{mathtools}
\usepackage{lipsum}
\usepackage{xfrac}
\usepackage{soul}
\usepackage{setspace}
\usepackage{float}
\usepackage{calc,ifthen}
\usepackage{epsfig}
\usepackage[normalem]{ulem} 
\usepackage{xr}
\bibliographystyle{pnas}

\usepackage{amssymb,amsfonts,amsmath}

% references
\usepackage[bibstyle=authoryear, citestyle=authoryear-comp,hyperref=auto]{biblatex}
\bibliography{references}

\definecolor{tangocolordarkchameleon}{HTML}{4E9A06}
\definecolor{tangocolordarkscarletred}{HTML}{CE5C00}
\definecolor{boiseblue}{RGB}{31,65,155}

\renewcommand{\topfraction}{1}             	% PLEASE DO NOT REMOVE THESE COMMANDS:
\renewcommand{\dbltopfraction}{1}          	% THE FIGURES GET PLACED AWFULLY WITHOUT
\renewcommand{\bottomfraction}{1}          % THEM!!
\renewcommand{\textfraction}{0}
\newcommand{\rep}[2]{\sout{#1} \textcolor{red}{#2}}

\nonstopmode

% document properties
\title{Is a double-stranded DNA the simplest \\ \vskip0.5ex and fastest biomolecular rotary motor}
\author{\textcolor{boiseblue}{\vskip0.5ex Franky Djutanta$^{a}$, Bernard Yurke$^{b}$ \textrm{\textit{\&}}  Rizal F. Hariadi$^{c}$\\
\vskip0.75ex {\small $^{a}$ The Biodesign Institute, Arizona State University, $^{b}$ Department of Physics, Boise State University,\\ $^{c}$Department of Physics \textrm{\textit{\&}} the Biodesign Institute, Arizona State University.}}}
\institute{\ }
%}}}

\begin{document}
\begin{frame}{}
\begin{columns}[t]

% ===== Column 1 ===== {{{
\begin{column}{.32\linewidth}
\vskip -0.5ex
\begin{block}{sMotivation and background} % {{{
		\begin{itemize}
            \item Molecular rotary motors are essential for living system. 
            \item For instance, F$_0$F$_1$-ATP synthase converts adenosine diphosphate (ADP) into adenosine triphosphate (ATP) by rotary motion.
		    \item However, to replicate a molecular rotary motor requires an uneasy process.
		    \item dsDNA may have a potential of becoming a molecular rotary motor because of its helix shape. Under a uniform field, the helix shape may have generated moment across the axis of dsDNA.
            \item The twist motion caused by the moment will be proportional to the voltage applied. 
            \item A verification by solving with Navier-Stokes equations
		\end{itemize}

\vskip0.5ex
\end{block} \vskip2ex % }}}

\begin{block}{Experimental Setup} % {{{
    

\end{block} % }}}

\end{column}
%}}}

% ===== Column 2 ===== {{{
\begin{column}{.32\linewidth}
\vskip -0.5ex
\begin{block}{Theoretical approach}    % Block 1
		\begin{itemize}
		\item Molecular motors are  
		\item The properties of the actomyosin interaction have mainly been examined at two levels: skinned muscle fibers and \\ single molecule measurements
		\item Using a DNA nanotube scaffold, we have engineered \\ artificial myosin filaments with defined organization. 		
		\item Using a DNA nanotube-based O-shaped Myosin gliding assay (O-Myo), we continuously monitored interactions between small myosin ensembles and single actin filaments.
			\begin{itemize}
				\item One step closer toward rigorous characterization of the lifetime of myosin motors.		\end{itemize}
		\end{itemize}
\vskip0.5ex
\end{block}
\end{column}    
%}}}

% ===== Column 3 ===== {{{
\begin{column}{.32\linewidth}
\vskip -0.5ex
\begin{block}{dsDNA Modification} % {{{
    

\end{block} \vskip2ex % }}}

\begin{block}{Questions} % {{{
		\begin{itemize}
		\item In muscle, contraction and force generation emerge from \\ the coordinated interactions between astronomical number \newline of myosin motors and actin filaments. 
		\item The properties of the actomyosin interaction have mainly been examined at two levels: skinned muscle fibers and \\ single molecule measurements
		\item Using a DNA nanotube scaffold, we have engineered \\ artificial myosin filaments with defined organization. 		
		\item Using a DNA nanotube-based O-shaped Myosin gliding assay (O-Myo), we continuously monitored interactions between small myosin ensembles and single actin filaments.
			\begin{itemize}
				\item One step closer toward rigorous characterization of the lifetime of myosin motors.		\end{itemize}
		\end{itemize}
\vskip0.5ex
\end{block} \vskip2ex % }}}

\begin{block}{References} % {{{

\vskip0.5ex
\end{block} % }}}

\end{column}
%}}}

\end{columns}
\end{frame}

\end{document}
